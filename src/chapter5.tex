\chapter{Conclusions}
\label{ch:conclusions}

Achieving consistency in large-scale distributed systems is not an easy task. To
make things more difficult, designers need to also ensure high-throughput,
low-latencies accesses to the databases. However, building reliable distributed
systems demands trade-offs between consistency and availability as stated by the
CAP theorem~\cite{Gilbert:2002:BCF:564585.564601}. Eventual consistency is a
technique of compromise, widely adopted, but lacking a rigorous theoretical
foundation which makes current approaches ad-hoc and
error-prone~\cite{DeCandia:2007:DAH:1294261.1294281}.

The concept of CRDTs defines replicated data types that have mathematical
properties conferring them a form of eventual consistency, strong eventual
consistency. This model can be described from two equivalent perspectives:
a) state-based: object replicas apply updates locally and later exchange and
merge their states, and b) operation-based: update operations are distributed
among replicas over a reliable broadcast communication channel. Both approaches
guarantee convergence towards a common state without application-level conflict
resolutions, roll-backs, or consensus among
replicas~\cite{Terry:1995:MUC:224056.224070}. These features come at the cost
of anomalies for some types discussed in Section~\ref{sec:examples_of_crdts},
which are in general acceptable and application specific. 

On the practical side, CRDTs can be deployed in various topologies since the
synchronization procedure between replicas is flexible: pulling the updates can
be done at any time and from any replica depending on requirements. Moreover,
they can be used in conjunction with existing distribution protocols, such as
gossip or anti-entropy~\cite{Demers:1987:EAR:41840.41841,
Petersen:1997:FUP:268998.266711}.

This thesis gave a theoretical background introduction on CRDTs and presented
the trade-offs for each of the two styles. A number of practical examples of
such types were given with a focus on the set container. The main contributions
included an algorithm for efficient delta-based synchronization, an extension to
the state-based set to support per-replica partitioning, and a garbage
collection mechanism to alleviate the problem of unbounded database growth.
To test these concepts in practice, the design and implementation of a client
library together with results on evaluating its performance were given.

Future work can be done to improve the implementation details of the client
library. One example is the dynamic estimation of the page size in the algorithm
for the \textit{merge} procedure from Listing~\ref{alg:merge} using heuristics
such as the elements in the current page. If one can analyse this series of
elements and infer a curve fitting formula for the timestamps, then a more
accurate estimation on the next page size can be made to increase the
probability of finding the required timestamp value.

A second improvement to the library concerns the garbage collection mechanism.
Currently, only elements are expired from the database, but the indices remain
untouched. One way to handle this is to run a periodic scan and to remove index
entries which have no corresponding elements. Alternatively, a Redis sorted set
can be used: for each update to the database, the id of the new element together
with its expiration time are inserted to this set. Since the set is sorted by
the expiration time, it is easy to find out which elements are expired and
remove their ids from the indices.

Lastly, given the modularity of CRDTs, one can build more complex structures by
composing simpler ones. Graphs, as seen in Section~\ref{sec:examples_of_crdts},
can be specified using two CRDT sets, one for vertices and another for edges.
Thus, SOR-Sets constitute important basic blocks in architectural design of
replicated databases. Employing graph-like types in such systems helps to
associate connexions between logically related elements from different sets
according to specific criteria and semantics, as discussed in
Section~\ref{sec:the_case_for_a_sor-set}. Another example is the design of a web
search engine, where a graph may be used to compute page
ranks~\cite{Shapiro:2011:CRD:2050613.2050642}. In this case, pages are stores as
vertices and links between them as edges.